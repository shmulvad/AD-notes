\documentclass[a4paper,11pt]{article}
% Various packages
\usepackage{siunitx}
\usepackage[utf8]{inputenc} % æøå
\usepackage[T1]{fontenc} % mere æøå
\usepackage[danish]{babel} % orddeling
\usepackage{verbatim} % så man kan skrive ren tekst
\usepackage{graphicx}
\graphicspath{{assets/}}
\usepackage{a4wide}
\usepackage{url}
\usepackage[left=2cm,top=2cm,bottom=1.5cm,right=2cm]{geometry}
\usepackage{amsmath}
\usepackage{amssymb}
\usepackage{amsthm}
\usepackage{wrapfig}
\usepackage{fixme}
\usepackage{color}
\usepackage{pstricks}
\usepackage{pdfpages} % include pdf
\usepackage{float} % Use [H] in figures
\usepackage{subcaption} % For subfigures
\usepackage{color} % May be necessary if you want to color links
\usepackage{hyperref} % Make references clickable
\usepackage[nameinlink,capitalize]{cleveref} % Make eq:refs be in style (1)
\usepackage[linesnumbered, commentsnumbered, lined, ruled, vlined,
%noend  % Have no ⌊-like symbol to indicate end of scope in pseudocode
]{algorithm2e} % Doc: https://goo.gl/6bC1qZ

% Ændr på navnene der vises når man bruger \autoref{label}
\def\sectionautorefname{Sektion}
\renewcommand{\equationautorefname}{Ligning}
\def\figureautorefname{Figur}
\AtBeginDocument{\renewcommand{\ref}[1]{\autoref{#1}}}

% Sæt \ref{} til at kalde \autoref{}
\AtBeginDocument{\renewcommand{\ref}[1]{\autoref{#1}}}

% Ændr ''*'' i math-felter til \cdot
\DeclareMathSymbol{*}{\mathbin}{symbols}{"01}

% Sæt farver for interne referencer og links
\definecolor{darkblue}{RGB}{25,25,112}
\hypersetup{
	colorlinks=true,    %set true if you want colored links
	linktoc=all,        %set to all if you want both sections and subsections linked
	linkcolor=darkblue, %choose some color if you want links to stand out
	filecolor=blue,     %
	citecolor=black,    %
	urlcolor=cyan,      %
}

% Set indentation to 0:
\setlength\parindent{0pt}

% Keywords relateret til algorithm2e pakken
\newcommand{\True}{\textbf{true}}\newcommand{\False}{\textbf{false}}
\SetStartEndCondition{ }{}{}%
\SetKwProg{Fn}{def}{\string:}{}
\SetKw{KwTo}{to}
\SetKwFor{For}{for}{}{}% 
\SetKwFor{ForEach}{foreach}{}{}% 
\SetKwIF{If}{ElseIf}{Else}{if}{}{elif}{else}{end}% 
\SetKwFor{While}{while}{}{end}\SetKwProg{Fn}{}{}{}
\SetKwInOut{Input}{input}\SetKwInOut{Output}{output}
\setlength{\algomargin}{3em}\DontPrintSemicolon

\newcommand{\longspace}{{\ \ \ \ \ \ \ \ \ \ \ \ \ \ }}
\renewcommand{\P}{{\mathbb P}}
\newcommand{\parfrac}[1]{\frac{\partial}{\partial #1}}
\renewcommand{\num}{{\textrm{num} }}
\newcommand{\size}{{\textrm{size} }}
\newcommand{\ift}{{\textrm{if } }}

% Dynamiske (), <>, ceil, floor
\newcommand{\p}[1]{\left( #1 \right)}
\newcommand{\pbig}[1]{\big( #1 \big)}
\newcommand{\pBig}[1]{\Big( #1 \Big)}
\newcommand{\pbigg}[1]{\bigg( #1 \bigg)}
\newcommand{\larr}[1]{\left< #1 \right>}
\newcommand{\ceil}[1]{\left\lceil #1 \right\rceil}
\newcommand{\floor}[1]{\left\lfloor #1 \right\rfloor}


% Squiggly arrows
\DeclareFontFamily{U} {MnSymbolC}{}

\DeclareFontShape{U}{MnSymbolC}{m}{n}{
	<-6> MnSymbolC5
	<6-7> MnSymbolC6
	<7-8> MnSymbolC7
	<8-9> MnSymbolC8
	<9-10> MnSymbolC9
	<10-12> MnSymbolC10
	<12-> MnSymbolC12}{}
\DeclareFontShape{U}{MnSymbolC}{b}{n}{
	<-6> MnSymbolC-Bold5
	<6-7> MnSymbolC-Bold6
	<7-8> MnSymbolC-Bold7
	<8-9> MnSymbolC-Bold8
	<9-10> MnSymbolC-Bold9
	<10-12> MnSymbolC-Bold10
	<12-> MnSymbolC-Bold12}{}

\DeclareSymbolFont{MnSyC} {U} {MnSymbolC}{m}{n}

\DeclareMathSymbol{\MNrhd}{\mathbin}{MnSyC}{76}
\DeclareMathSymbol{\MNlhd}{\mathbin}{MnSyC}{78}
% =============================================
\DeclareFontFamily{U} {MnSymbolD}{}

\DeclareFontShape{U}{MnSymbolD}{m}{n}{
	<-6> MnSymbolD5
	<6-7> MnSymbolD6
	<7-8> MnSymbolD7
	<8-9> MnSymbolD8
	<9-10> MnSymbolD9
	<10-12> MnSymbolD10
	<12-> MnSymbolD12}{}
\DeclareFontShape{U}{MnSymbolD}{b}{n}{
	<-6> MnSymbolD-Bold5
	<6-7> MnSymbolD-Bold6
	<7-8> MnSymbolD-Bold7
	<8-9> MnSymbolD-Bold8
	<9-10> MnSymbolD-Bold9
	<10-12> MnSymbolD-Bold10
	<12-> MnSymbolD-Bold12}{}

\DeclareSymbolFont{MnSyD} {U} {MnSymbolD}{m}{n}
\DeclareMathSymbol{\MNsim}{\mathbin}{MnSyD}{2}

% =============================================

\usepackage{amssymb,amsmath,stackengine}
\stackMath
\newcommand\rsquigarrow[1]{%
	\mathbin{\stackon[2pt]{\rightsquigarrow}{\scriptscriptstyle #1 }}
}

\author{Søren Mulvad, rbn601}
